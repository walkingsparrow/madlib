\documentclass[letterpaper,11pt]{scrreprt}
\usepackage[american]{babel}
\usepackage[latin1]{inputenc}
\usepackage[T1]{fontenc}
\usepackage[top=1in,bottom=1in,left=1in,right=1in]{geometry}
\usepackage{lmodern}

\usepackage{amsmath}
\usepackage{amssymb}

\usepackage[hyperref]{ntheorem}
\usepackage[
	bookmarks,
	colorlinks=false,
	linkcolor=blue,
	citecolor=blue,
	pagebackref=false,
	pdftitle={MADlib Design Document},
	pdfauthor={Florian Schoppmann},
	pdfsubject={},
	pdfkeywords={}
]{hyperref}
\usepackage{csquotes}                  % Strongly recommended for biblatex
\usepackage[
	backend=bibtex,
	maxnames=2,
	maxbibnames=20,
	firstinits=true
]{biblatex}
\usepackage{scrpage2}                  % Headers and footers
\usepackage{color}                     % Colors, possibly only for \todo
\usepackage{enumitem}                  % enumerate environment
\usepackage{ctable}
\usepackage{tabularx}
\usepackage{xspace}                    % Correct spaces after \newcommand definitions
\usepackage[noend]{algpseudocode}      % algorithm environment
\usepackage{listings}                  % Code snippets
\usepackage{bbding}
\usepackage{latex/tikz-uml}            % UML diagrams

% BEGIN Doc Layout
	\allowdisplaybreaks[3]

	\pagestyle{scrheadings}
	\automark[chapter]{section}

	\setkomafont{disposition}{\normalcolor\bfseries}
	\setkomafont{descriptionlabel}{\bfseries}
	\setkomafont{captionlabel}{\usekomafont{disposition}}

	\setlength{\arrayrulewidth}{.5pt}
	\numberwithin{equation}{section}
	\renewcommand{\theenumi}{\roman{enumi}}
	\renewcommand{\labelenumi}{\theenumi)}

	\newcommand{\otoprule}{\midrule[\heavyrulewidth]}

	\setcounter{secnumdepth}{3}

	\makeatletter
	% Algorithms are expected to have an optional argument of form
	% FunctionName$(ArgumentList)$, e.g., DiscreteSample$(A, w)$
	\def\internal@funcName#1$(#2)${#1}
	\newcommand\funcName[1]{\internal@funcName #1}
	\newtheoremstyle{algorithm}
		{\item[\rlap{\vbox{\hbox{\hskip\labelsep \theorem@headerfont
			##1\ ##2\theorem@separator}\hbox{\strut}}}]}%
		{\item[\rlap{\vbox{\hbox{\hskip\labelsep {\theorem@headerfont
			##1}\ \normalfont\texttt{##3}{\theorem@headerfont\theorem@separator}}\hbox{\strut}}}]%
			\def\@currentlabel{\texttt{\funcName{##3}}}}
	\makeatother

	\makeatletter
	% Also display JSTOR in small caps
	% http://sourceforge.net/tracker/index.php?func=detail&aid=3152938&group_id=244752&atid=1126006
	\DeclareFieldFormat{eprint:arxiv}{%
	  \textsc{arXiv}\addcolon
	  \ifhyperref
	    {\href{http://arxiv.org/\abx@arxivpath/#1}{%
	       \nolinkurl{#1}%
	       \iffieldundef{eprintclass}
		 {}
		 {\addspace\texttt{\mkbibbrackets{\thefield{eprintclass}}}}}}
	    {\nolinkurl{#1}
	     \iffieldundef{eprintclass}
	       {}
	       {\addspace\texttt{\mkbibbrackets{\thefield{eprintclass}}}}}}
	\DeclareFieldFormat{eprint:jstor}{%
	  \mkbibacro{JSTOR}\addcolon\space
	  \ifhyperref
	    {\href{http://www.jstor.org/stable/#1}{\nolinkurl{#1}}}
	    {\nolinkurl{#1}}}
	% Some conferences do not have DOIs for their papers, but they do get
	% IDs in the ACM Digital Library. E.g., SODA papers.
	\DeclareFieldFormat{eprint:acm}{%
	  \mkbibacro{ACM}\addcolon\space
	  \ifhyperref
	    {\href{http://dl.acm.org/citation.cfm?id=#1}{\nolinkurl{#1}}}
	    {\nolinkurl{#1}}}
	\makeatother
	
	\newlist{moduleinfo}{description}{2}
	\setlist[moduleinfo]{style=multiline,labelindent=\leftmargini,leftmargin=3cm,rightmargin=\leftmargini,font=\bfseries}
	\newlist{modulehistory}{description}{2}
	\setlist[modulehistory]{style=multiline,leftmargin=1.1cm}
% END Doc Layout

% BEGIN General Definitions
	\newcommand{\todo}[1]{\textbf{\color{red}#1}}

	\newcommand{\specialcell}[3][t]{%
		\begin{tabular}[#1]{@{}#2@{}}#3\end{tabular}}

	% BEGIN Mathematical Definition
		% Space (only) in displaymath (e.g., between mathematical expression and punctuation mark)
		\newcommand{\SiM}{\mathchoice{\,}{}{}{}}
	% END Mathematical Operators

	% BEGIN URLs
		\newcommand{\mailto}[1]{\href{mailto:#1}{\nolinkurl{#1}}}
		\newcommand{\doi}[1]{DOI: \href{http://dx.doi.org/#1}{\nolinkurl{#1}}}
	% END URLs

	\makeatletter
	% BEGIN Mathematical Definitions
		% BEGIN Set Symbols
			\newcommand{\setsymbol}[1]{\mathbb{#1}}
			\newcommand{\N}{\@ifstar{\setsymbol{N}_0}{\setsymbol{N}}}
			\newcommand{\R}{\setsymbol{R}}
		    \newcommand{\Nupto}{\@ifstar{\Nupto@star}{\Nupto@nostar}}
		    \newcommand{\Nupto@star}[1]{[#1]_0}
		    \newcommand{\Nupto@nostar}[1]{[#1]}
		% END Set Symbols
		\renewcommand{\vec}[1]{\ensuremath{\boldsymbol{#1}}}
	% END Mathematical Definitions
	\makeatother

	\renewcommand{\vec}[1]{\ensuremath{\boldsymbol{#1}}}
	\newcommand{\enumref}[1]{(\ref{#1})}

	\makeatletter
	\newcommand{\symlabel}[2]{\def\@currentlabel{\texttt{#1}}\texttt{#1}\label{#2}}
	\makeatother
	
	\newcommand{\Warning}[1]{\marginpar[\HandRight]{\HandLeft}\textbf{#1}}

	% BEGIN Algorithms
	\theoremstyle{algorithm}
	\theorembodyfont{\upshape}
	\newtheorem{algorithm}{Algorithm}[section]

	\newlength{\alglabelwidth}
	\newcommand{\alginput}[1]{%
		\par\noindent%
		\settowidth{\alglabelwidth}{\emph{Output:}}%
		\makebox[\alglabelwidth][l]{\emph{Input:}} \begin{tabular}[t]{l} #1 \end{tabular}}
	\newcommand{\algoutput}[1]{%
		\par\noindent%
		\settowidth{\alglabelwidth}{\emph{Output:}}%
		\makebox[\alglabelwidth][l]{\emph{Output:}} \begin{tabular}[t]{l} #1 \end{tabular}}
	\newcommand{\algprecond}[1]{%
		\par\noindent\textit{Initialization/Precondition: #1}}

	\newcommand{\set}{\leftarrow}
	\DeclareMathOperator{\random}{random}
	\newcommand{\dist}{\ensuremath{\mathit{dist}}}
	\newcommand{\List}{\mathrm{List}}
	\newcommand{\Sample}{\mathit{Sample}}
	\algblockdefx[With]{With}{EndWith}%
		[1]{\textbf{with} #1 \textbf{do}}%
		[0]{End}
	\algnotext[With]{EndWith}
	% END Algorithms

	% BEGIN lstlisting environments
	\lstset{
		basicstyle=\ttfamily\footnotesize,       % the size of the fonts that are used for the code
		numbers=left,                   % where to put the line-numbers
		numberblanklines=false
		numbersep=1em,                  % how far the line-numbers are from the code
		basewidth=0.52em,
		tabsize=4,  		% sets default tabsize to 2 spaces
		xleftmargin=\leftmargini
	}
	\renewcommand*\thelstnumber{\the\value{lstnumber}:}
	% END lstlisting environments
	
	\lstnewenvironment{sql}[1][]{\lstset{language=SQL,gobble=4,emphstyle=\textit,#1}}{}
	\lstnewenvironment{cpp}[1][]{\lstset{language=C++,gobble=4,emphstyle=\textit,#1}}{}
	\lstnewenvironment{cppsnippet}[1][]{\lstset{basicstyle=\ttfamily,language=C++,stepnumber=0,gobble=8,emphstyle=\textit,xleftmargin=0pt,#1}}{}
% END General Definitions

\bibliography{../literature.bib}

% BEGIN Preamble
\title{%
	MADlib Design Document%
}

\begin{document}

\maketitle

\tableofcontents

\input{other-chapters/abstraction-layers}
\input{modules/sampling}
\input{modules/matrix-operations}
\input{modules/k-means}
\input{modules/low-rank-matrix-decomposition}
% When using TeXShop on the Mac, let it know the root document.
% The following must be one of the first 20 lines.
% !TEX root = ../design.tex

\chapter[Convex Programming Framework]{Convex Programming Framework}

\begin{moduleinfo}
\item[Author] \href{mailto:xfeng@cs.wisc.edu}{Xixuan (Aaron) Feng}
\item[History]
	\begin{modulehistory}
		\item[v0.5] Initial revision
	\end{modulehistory}
\end{moduleinfo}

% Motivation. Why do we want to have this abstract layer?
The nature of MADlib drives itself to support many different kinds of data modeling modules, such as logistic regression, support vector machine, matrix factorization, etc.
However, keeping up with the state of the art and experimenting with individual data modeling modules require significant development and quality-assurance effort.
Therefore, to lower the bar of adding and maintaining new modules, it is crucial to identify the invariants among many important modules, in turn, abstract and encapsulate them as reusable components.

Bismarck \cite{DBLP:conf/sigmod/FengKRR12} is such a unified framework that links many useful statistical modeling modules and the relational DBMS, by introducing a well-studied formulation, convex programming, in between.
Incremental Gradient Descent (IGD) has also been shown as a very effective algorithm to solve convex programs in the relational DBMS environment.
But it is natural that IGD does not always fit the need of MADlib users who are applying convex statistical modeling to various domains.
Driven by this, convex programming framework in MADlib also implements other algorithms that solves convex programs, such as Newton's method and conjugate gradient methods.

\section{Introduction}
% Problem definition. What are the problems that we can solve, formally and example applications?
% linearly separable, unconstrained, continuous, deterministic, convex, minimization problems.
This section is to first explain, formally, the type of problems that we consider in the MADlib convex programming framework, and then give a few example modules.

\subsection{Formulation}
We support numerical optimization problems with an objective function that is a sum of many component functions \cite{springerlink:10.1007/s10107-011-0472-0}, such as
\[\min_{w \in \mathbb{R}^N} \sum_{m=1}^M f_{z_m}(w),\]
where $z_m \in \mathcal{O}, m = 1,...,M$, are observations, and $f_{z_m} : \mathbb{R}^N \to \mathbb{R}$ are convex functions.
For simplicity, let $z_{1:M}$ denote $\{z_m \in \mathcal{O} | m = 1,...,M\}$.
Note: given $z_{1:M}$, let $F(w) = \sum_{m=1}^M f_{z_m}(w)$, and $F : \mathbb{R}^N \to \mathbb{R}$ is also convex.

\subsection{Examples}
Many popular models can be formulated in the above form, with $f_{z_m}$ being properly specified.

\paragraph{Logistic Regression.} The component function is given by
\[f_{(x_m, y_m)}(w) = \log(1 + e^{- y_m w^{T} x_m}),\]
where $x_m \in \mathbb{R}^N$ are values of independent variables, and $y_m \in \{-1, 1\}$ are values of the dependent variable, $m = 1,...,M$.

\paragraph{Linear SVM with hinge loss.} The component function is given by
\[f_{(x_m, y_m)}(w) = \max(0, 1 - y_m w^{T} x_m),\]
where $x_m \in \mathbb{R}^N$ are values of features, and $y_m \in \{-1, 1\}$ are values of the label, $m = 1,...,M$.
Bertsekas \cite{springerlink:10.1007/s10107-011-0472-0} gives many other examples across application domains.

\section{Algorithms}
\paragraph{Gradient Descent.}
A most-frequently-mentioned algorithm that solves convex programs is gradient descent.
This is an iterative algorithm and the iteration is given by
\[w_{k+1} = w_k - \alpha_k \nabla F(w_k),\]
where, given $z_{1:M}$, $F(w) = \sum_{m=1}^M f_{z_m}(w)$, and $\alpha_k$ is a positive scalar, called stepsize (or step length).
Gradient descent algorithm is simple but usually recognized as a slow algorithm with linear convergence rate, while other algorithms like conjugate gradient methods and Newton's method has super-linear convergence rates \cite{nocedal2006numerical}.

\paragraph{Line Search: A Class of Algorithms.}
% line search & trust region
Convex programming has been well studied in the past few decades, and two main classes of algorithms are widely considered: line search and trust region (\cite{nocedal2006numerical}, section 2.2).
Because line search is more commonly deployed and discussed, we focus on line search in MADlib, although some of the algorithms we discuss in this section can also easily be formulated as trust region strategy.
% general form of line search: w += \alpha p_k
All algorithms of line search strategies have the iteration given by
\[w_{k+1} = w_k + \alpha_k p_k,\]
where $p_k \in \mathbb{R}^N$ is search direction, and stepsize $\alpha_k$ \cite{nocedal2006numerical}.
Specifiedly, for gradient descent, $p_k$ is the steepest descent direction $- \nabla \sum_{m=1}^M f_{z_m}(w_k)$.

\subsection{Formal Description of Line Search}
\begin{algorithm}[line-search$(z_{1:M})$] \label{alg:line-search}
\alginput{Observation set $z_{1:M}$,\\
convergence criterion $\mathit{Convergence}()$,\\
start strategy $\mathit{Start}()$,\\
initialization strategy $\mathit{Initialization}()$,\\
transition strategy $\mathit{Transition}()$,\\
finalization strategy $\mathit{Finalization}()$}
\algoutput{Coefficients $w \in \mathbb{R}^N$}
\algprecond{$iteration = 0, k = 0$}
\begin{algorithmic}[1]
	\State $w_\text{new} \set \mathit{Start}(z_{1:M})$
	\Repeat
        \State $w_\text{old} \set w_\text{new}$
        \State $\mathit{state} \set \mathit{Initialization}(w_\text{new})$
		\For{$m \in 1..M$} \Comment{Single entry in the observation set}
			\State $\mathit{state} \set \mathit{Transition}(\mathit{state}, z_m)$
                \Comment{Usually computing derivative}
		\EndFor
		\State $w_\text{new} \set Finalization(\mathit{state})$
	\Until{$Convergence(w_\text{old}, w_\text{new}, \mathit{iteration})$}
    \State \Return $w_\text{new}$
\end{algorithmic}
\end{algorithm}

\paragraph{Programming Model.}
We above give the algorithm of generic line search strategy, in the fashion of the selected programming model supported by MADlib (mainly user-defined aggregate).

\paragraph{Parallelism.}
The outer loop is inherently sequential.
We require the inner loop is data-parallel.
Simple component-wise addition or model averaging \cite{DBLP:conf/nips/DuchiAW10} are used to merge two states.
A merge function is not explicitly added to the pseudocode for simplicity.
A separate discussion will be made when necessary.

\paragraph{Convergence criterion.}
Usually, following conditions are combined by AND, OR, or NOT.
\begin{enumerate}
    \item The change in the objective drops below a given threshold (E.g., negative log-likelihood, root-mean-square error).
    \item The value of the objective matches some pre-computed value.
    \item The maximum number of iterations has been reached.
    \item There could be more.
\end{enumerate}
In MADlib implementation, the computation of objective is paired up with line-search to share data I/O.

\paragraph{Start strategy.}
In most cases, zeros are used unless otherwise specified.

\paragraph{Transition and finalization strategies.}
The coefficients update code ($w_{k+1} \set w_k + \alpha_k p_k$) is put into either $\mathit{Transition}()$ or $\mathit{Finalization}()$.
These two functions contain most of the computation logic, for computing the search direction $p_k$.
We discuss details of individual algorithms in the following sections.
For simplicity, global iterator $k$ is read and updated in place by these functions without specifed as an additional argument.

\subsection{Incremental Gradient Descent (IGD)}
% motivation and introduction of IGD
A main challenge arises when we are handling large amount of data, $M \gg 1$, where the computation of $\nabla (\sum_{m=1}^M f_{z_m})$ requires a whole pass of the observation data which is usually expensive.
What distinguishes IGD from other algorithms is that it approximates  $\nabla (\sum_{m=1}^M f_{z_m}) = \sum_{m=1}^M (\nabla f_{z_m})$ by the gradient of a single component function $\nabla f_{z_m}$
\footnote{$z_m$ is usually selected in a stochastic fashion.
Therefore, IGD is also referred to as stochastic gradient descent.
The convergence and convergence rate of IGD are well developed \cite{springerlink:10.1007/s10107-011-0472-0}, and IGD is often considered to be very effective with $M$ being very large \cite{DBLP:conf/nips/BottouB07}.}.
The reflection of this to the pseudocode makes the coefficients update code ($w_{k+1} \set w_k + \alpha_k p_k$) in $\mathit{Transition}()$ instead of in $\mathit{Finalization}()$.

\subsubsection{Initialization Strategy}
\begin{algorithm}[initialization-igd$(w)$] \label{alg:initialization-igd}
\alginput{Coefficients $w \in \mathbb{R}^N$}
\algoutput{Transition state $\mathit{state}$}
\begin{algorithmic}[1]
    \State $\mathit{state}.w_k \set w$
    \State \Return $\mathit{state}$
\end{algorithmic}
\end{algorithm}

\subsubsection{Transition Strategy}
\begin{algorithm}[transition-igd$(\mathit{state}, z_m)$] \label{alg:transition-igd}
\alginput{Transition state $\mathit{state}$,\\
observation entry $z_m$,\\
stepsize $\alpha \in \mathbb{R}^+$,\\
gradient function $\mathit{Gradient}()$}
\algoutput{Transition state $\mathit{state}$}
\begin{algorithmic}[1]
    \State $p_k \set - \mathit{Gradient}(\mathit{state}.w_k, z_m)$
        \Comment{Previously mentioned as $p_k = - \nabla f_{z_m}$}
    \State $\mathit{state}.w_{k+1} \set \mathit{state}.w_k + \alpha p_k$
    \State $k \set k + 1$
        \Comment{In-place update of the global iterator}
    \State \Return $\mathit{state}$
\end{algorithmic}
\end{algorithm}

\paragraph{Stepsize.}
In MADlib, we support only constant stepsize for simplicity.
Although IGD with constant stepsizes does not even have convergence guarantee \cite{springerlink:10.1007/s10107-011-0472-0}, but it works reasonably well in practice so far \cite{DBLP:conf/sigmod/FengKRR12} with some proper tuning.
Commonly-used algorithms to tune stepsize (\cite{bertsekas1999nonlinear}, appendix C) are mostly heuristics and do not have strong guarantees on convergence rate.
More importantly, these algorithms require many evaluations of the objective function, which is usually very costly in use cases of MADlib.

\paragraph{Gradient function.}
A function where IGD accepts computational logic of specified modules.
In MADlib convex programming framework, currently, there is no support of objective functions that does not have gradient or subgradient.
Those objective functions that is not linearly separable is not currently supported by the convex programming framework, such as Cox proportional hazards models \cite{Cox1972}.

\subsubsection{Finalization Strategy}
\begin{algorithm}[finalization-igd$(\mathit{state})$] \label{alg:finalization-igd}
\alginput{Transition state $\mathit{state}$}
\algoutput{Coefficients $w \in \mathbb{R}^N$}
\begin{algorithmic}[1]
    \State \Return $\mathit{state}.w_k$
        \Comment{Trivially return $w_k$}
\end{algorithmic}
\end{algorithm}

\subsection{Conjugate Gradient Methods}
% Simple description of conjugate gradient
Conjugate gradient methods that solve convex programs are usually refered to as nonlinear conjugate gradient mthods.
The key difference between conjugate gradient methods and gradient descent is that conjuagte gradient methods perform adjustment of the search direction $p_k$ by considering gradient directions of previous iterations in some intriguing way.
We skip the formal desciption of conjugate gradient methods that can be found in the references (such as Nocedal \& Wright \cite{nocedal2006numerical}, section 5.2). 

\subsubsection{Initialization Strategy}
\begin{algorithm}[initialization-cg$(w)$] \label{alg:initialization-cg}
\alginput{Coefficients $w \in \mathbb{R}^N$,\\
gradient value $g \in \mathbb{R}^N$ (i.e., $\sum_{m=1}^M \nabla f_{z_m}(w_{k-1})$),\\
previous search direction $p \in \mathbb{R}^N$}
\algoutput{Transition state $\mathit{state}$}
\begin{algorithmic}[1]
    \State $\mathit{state}.p_{k-1} \set p$
    \State $\mathit{state}.g_{k-1} \set g$
    \State $\mathit{state}.w_k \set w$
    \State $\mathit{state}.g_k \set 0$
    \State \Return $\mathit{state}$
\end{algorithmic}
\end{algorithm}

\subsubsection{Transition Strategy}
\begin{algorithm}[transition-cg$(\mathit{state}, z_m)$] \label{alg:transition-cg}
\alginput{Transition state $\mathit{state}$,\\
observation entry $z_m$,\\
gradient function $\mathit{Gradient}()$}
\algoutput{Transition state $\mathit{state}$}
\begin{algorithmic}[1]
    \State $\mathit{state}.g_k \set \mathit{state}.g_k + \mathit{Gradient}(\mathit{state}.w_k, z_m)$
    \State \Return $\mathit{state}$
\end{algorithmic}
\end{algorithm}

\subsubsection{Finalization Strategy}
\begin{algorithm}[finalization-cg$(\mathit{state})$] \label{alg:finalization-cg}
\alginput{Transition state $\mathit{state}$,\\
stepsize $\alpha \in \mathbb{R}^+$,\\
update parameter strategy $\mathit{Beta}()$}
\algoutput{Coefficients $w \in \mathbb{R}^N$,\\
gradient value $g \in \mathbb{R}^N$ (i.e., $\sum_{m=1}^M \nabla f_{z_m}(w_{k-1})$),\\
previous search direction $p \in \mathbb{R}^N$}
\begin{algorithmic}[1]
    \If{k = 0}
        \State $\mathit{state}.p_k \set - \mathit{state}.g_k$
    \Else
        \State $\beta_k \set \mathit{Beta}(state)$
        \State $p_k \set - \mathit{state}.g_k + \beta_k p_{k-1}$
    \EndIf
    \State $\mathit{state}.w_{k+1} \set \mathit{state}.w_k + \alpha p_k$
    \State $k \set k + 1$
    \State $p \set p_{k-1}$
        \Comment{Implicitly returning}
    \State $g \set \mathit{state}.g_{k-1}$
        \Comment{Implicitly returning again}
    \State \Return $\mathit{state}.w_k$
\end{algorithmic}
\end{algorithm}

\paragraph{Update parameter strategy.}
For cases that $F$ is strongly convex quadratic (e.g., ordinary least squares), $\beta_k$ can be computed in closed form, having $p_k$ be in conjugate direction of $p_0,...,p_{k-1}$.
For more general objective functions, many different choices of update parameter are proposed \cite{hager2006survey, nocedal2006numerical}, such as
\[\beta_k^{FR} = \frac{\|g_k\|^2}{\|g_{k-1}\|^2},\]
\[\beta_k^{HS} = \frac{g_k^T (g_k - g_{k-1})}{p_{k-1}^T (g_k - g_{k-1})},\]
\[\beta_k^{PR} = \frac{g_k^T (g_k - g_{k-1})}{\|g_{k-1}\|^2},\]
\[\beta_k^{DY} = \frac{\|g_k\|^2}{p_{k-1}^T (g_k - g_{k-1})},\]
where $g_k = \sum_{m=1}^M \nabla f_{z_m}(w_k)$, and $p_k = - g_k + \beta_k p_{k-1}$.
We choose the strategy proposed by Dai and Yuan due to lack of mechanism for stepsize tuning in MADlib, which is required for other alternatives to guarantee convergence rate. (See Theorem 4.1 in Hager and Zhang \cite{hager2006survey}).
In fact, lack of sufficient stepsize tuning for each iteration individually could make conjugate gradient methods have similar or even worse convergence rate than gradient descent.
This should be fixed in the future.

\subsection{Newton's Method}
Newton's method uses a search direction other than the steepest descent direction -- \emph{Newton direction}.
The Newton direction is very effective in the cases that the objective function is not too different from a quadratic approximation, and it gives quadratic convergence rate by considering Taylor's theorem.
Formally, the Newton direction is given by
\[p_k = -(\nabla^2 F(w_k))^{-1} \nabla F(w_k),\]
where, given $z_{1:M}$, $F(w) = \sum_{m=1}^M f_{z_m}(w)$, and $H_k = \nabla^2 F(w_k)$ is called the Hessian matrix.

\subsubsection{Initialization Strategy}
\begin{algorithm}[initialization-newton$(w)$] \label{alg:initialization-newton}
\alginput{Coefficients $w \in \mathbb{R}^N$}
\algoutput{Transition state $\mathit{state}$}
\begin{algorithmic}[1]
    \State $\mathit{state}.w_k \set w$
    \State $\mathit{state}.g_k \set 0$
    \State $\mathit{state}.H_k \set 0$
    \State \Return $\mathit{state}$
\end{algorithmic}
\end{algorithm}

\subsubsection{Transition Strategy}
\begin{algorithm}[transition-newton$(\mathit{state}, z_m)$] \label{alg:transition-newton}
\alginput{Transition state $\mathit{state}$,\\
observation entry $z_m$,\\
gradient function $\mathit{Gradient}()$,\\
Hessian matrix function $\mathit{Hessian}()$}
\algoutput{Transition state $\mathit{state}$}
\begin{algorithmic}[1]
    \State $\mathit{state}.g_k \set \mathit{state}.g_k + \mathit{Gradient}(\mathit{state}.w_k, z_m)$
    \State $\mathit{state}.H_k \set \mathit{state}.H_k + \mathit{Hessian}(\mathit{state}.w_k, z_m)$
    \State \Return $\mathit{state}$
\end{algorithmic}
\end{algorithm}

\subsubsection{Finalization Strategy}
\begin{algorithm}[finalization-newton$(\mathit{state})$] \label{alg:finalization-newton}
\alginput{Transition state $\mathit{state}$}
\algoutput{Coefficients $w \in \mathbb{R}^N$}
\begin{algorithmic}[1]
    \State $p_k \set - (\mathit{state}.H_k)^{-1} \mathit{state}.g_k$
    \State $\mathit{state}.w_{k+1} \set \mathit{state}.w_k + p_k$
    \State $k \set k + 1$
    \State \Return $\mathit{state}.w_k$
\end{algorithmic}
\end{algorithm}

\paragraph{Hessian Matrix Function.}
A function where Newton's method accepts another computational logic of specified modules. See also gradient function.

\paragraph{Inverse of the Hessian Matrix.}
The inverse of Hessian matrix may not always exist if the Hessian is not guaranteed to be positive definite ($\nabla^2 F = 0$ when $F$ is linear).
We currently only support Newton's method for objetcive functions that is strongly convex.
This may sometimes mean an objective function that is not globally strongly convex but Newton's method works well with a good starting point as long as the objective function is strongly convex in a convex set that contains the given starting point and the minimum.
A few techniques that modify the Newton's method to adapt objective functions that are not strongly convex can be found in the references \cite{bertsekas1999nonlinear, nocedal2006numerical}.

\paragraph{Feed a Good Start Point.}
Since Newton's method is sensitive to the start point $w_0$, we provide a start strategy $\mathit{Start()}$ to accept a start point that may not be zeros.
It may come from results of other algorithms, e.g., IGD.

\subsection{Quasi-Newton Method}

\section{Implemented Machine Learning Algorithms}

We have implemented several machine learning algorithms under the
framework of convex optimization.

\subsection{Linear Ridge Regression}
Ridge regression is the most commonly used method of regularization of
ill-posed problems. Mathematically, it seeks to minimize 
\begin{equation}
Q\left(\vec{w},w_0;\lambda\right)\equiv \min_{\vec{w},w_0}\left[ \frac{1}{2N} \sum_{i=1}^{N} \left( y_i - w_0 -
    \vec{w} \cdot \vec{x}_i \right)^2
  +\frac{\lambda}{2}\|\vec{w}\|_2^2 \right]\ ,
\end{equation}
for a given value of $\lambda$, where $\vec{w}$ and $w_0$ are the fitting coefficients, and $\lambda$
is a non-negative regularization parameter. $\vec{w}$ is a vector in
$d$ dimensional space, and
\begin{equation}
\|\vec{w}\|_2^2 = \sum_{j=1}^{d}w_j^2 = \vec{w}^T\vec{w}\ .
\end{equation} 
When $\lambda = 0$, $Q$ is
the mean squared error of the fitting. 

The intercept term $w_0$ is not regularized, because this term is
fully decided by the mean values of $y_i$ and $\vec{x}_i$ and the
values of $\vec{w}$, and does not affect the model's complexity.

$Q\left(\vec{w},w_0;\lambda)$ is a quadratic function of $\vec{w}$ and
  $w_0$, and thus can be solved analytically
\begin{equation}
\vec{w}_{ridge}=\left(\lambda\vec{I}_d +
  \vec{X}^T\vec{X}\right)^{-1}\vec{X}^T\vec{y}\ .
\end{equation}
By using the available Newton method (Sec. 6.2.4), the above quantity can be easily
calculated from one single step of the Newton method.

Many packages for Ridge regularization actually regularize the fitting
coefficients not for the fitting model for the original data but for
the data that has be scaled. MADlib also provides this option. When
the normalization parameter is set to be True, which is by default
False, the data will be first converted to the following before
applying the Ridge regularization.
\begin{equation}
  x_i' \leftarrow \frac{x_i - \langle x_i \rangle}{\langle (x_i -
    \langle x_i \rangle)^2\rangle} \ ,
\end{equation}
\begin{equation}
y_i \leftarrow y_i - \langle y_i \rangle \ ,
\end{equation}
where $\langle \cdot \rangle = \sum_{i=1}^{N} \cdot / N$.

Note that Ridge regressions for scaled data and un-scaled data are not equivalent.
\input{modules/crf}
% When using TeXShop on the Mac, let it know the root document.
% The following must be one of the first 20 lines.
% !TEX root = ../design.tex

\chapter{Latent Dirichlet Allocation (LDA)}

\begin{moduleinfo}
\item[Author] \href{mailto:shengwen.yang@emc.com}{Shengwen Yang} (version 0.6 only)
\item[History]
    \begin{modulehistory}
        \item[v0.6] Initial revision of design document, complete rewrite of module, standard parallel implementation (support for local model update tuple by tuple and global model update iteration by iteration) 
        \item[v0.1] Initial version (An approximated implementation which has memory problem for big datasets)
    \end{modulehistory}
\end{moduleinfo}

LDA is a very popular technique for topic modeling. This module implements a parallel Gibbs sampling algorithm for LDA inference.

\section{Overview of LDA}
LDA\cite{Blei:2003:LDA:944919.944937} is a very popular technique for discovering the main themes or
topics from a large collection of unstructured documents and has been widely
applied to various fields, including text mining, computer vision, finance,
bioinformatics, cognitive science, music, and social sciences. 

With LDA, a document can be represented as a random mixture of latent topics,
where each topic can be characterized by a probability distribution over a
vocabulary of words. Given a large text corpus, LDA will be able to infer a set
of latent topics from the corpus, each represented with a multinomial
distribution over words, denoted as $P(w|z)$, and infer the topic distribution
for each document, represented as a multinomial distribution over topics,
denoted as $P(z|d)$.

Several methods have been proposed for the inference of LDA, including
variational Bayesian, expectation propagation, and Gibbs
sampling\cite{griffiths04finding}. Among of these methods, Gibbs sampling is
the most widely used one because it is simple, fast, and has very few
adjustable parameters. Besides, Gibbs sampling is easy to parallelize and easy
to scale up, which allows us to utilize a cluster of machines to deal with very
big datasets.

\section{Gibbs Sampling for LDA}
\subsection{Overview}
Althouth the derivation of Gibbs sampling for LDA is complicated, the results are very simple. The following equation tells us how to sample a new topic for a word in a corpus: 

\begin{equation}
P(z_i=k|{\bf z}_{-i},{\bf w}) \propto \frac{nwz_{-i,k}^{(w_i)} +\beta}{nz_{-i,k}+W \beta}\times(ndz_{-i,k}^{(d_i)}+\alpha)
\end{equation}
where:
\begin{itemize}
\item $i$ - index of word in the corpus
\item $d_i$ - docid of the $i_{th}$ word
\item $w_i$ - wordid of the $i_{th}$ word
\item $k$ - the $k_{th}$ topic, where $1 <= k <= T$, and $T$ is the number of topics
\item $z_i$ - topic assignment for the $i_{th}$ word
\item ${\bf z}_{-i}$ - topic assignments for other words excluding the $i_{th}$ word
\item ${\bf w}$ - all words in the corpus
\item $ndz$ - per-document topic counts
\item $nwz$ - per-word topic counts 
\item $nz$ - corpus-level topic counts
\end{itemize}

According to this equation, we can update the topic assignment to each word sequnetially. This process can be iterated enough times until the conditional distribution reachs a stable state.

\subsection{Parallization}
The parallization of the above algirhtm is very straightforward. The basic idea is to distribute a large set of documents to a cluster of segment nodes and allow each segment node to do Gibbs sampling on a subset of documents locally. Note that at the end of each iteration, the local models generated on each segment node will be merged to generate a global model, which will be distributed to each segment node at the begining of next iteration. 

Refer to \cite{Wang:2009:PPL:1574036.1574062} for a similar parallel implementation based on MPI and MapReduce.

\subsection{Formal Description}
\begin{algorithm}[gibbs-lda$(D, T, \alpha, \beta)$] \label{alg:gibbs-lda}

\alginput{
\\Dataset $D$,
\\topic number $T$,
\\prior on per-document topic distribution  $\alpha$,
\\prior on per-word topic distribution $\beta$
}

\algoutput{
\\Per-document topic distribution $P(z|d)$,
\\per-word topic distribution $P(z|w)$
}

\begin{algorithmic}[1]
    \State $ndz \set 0$
    \State $nwz \set 0$
    \State $nz \set 0$  
    \For{$d \in D$}
        \For{$w \in W_d$}
            \State $z \set \text{random}(T)$
            \State $ndz[d,z] \set ndz[d,z] + 1$
            \State $nwz[w,z] \set nwz[w,z] + 1$
            \State $nz[z] \set nz[z] + 1$
            \State $Z[d,w] \set z$
        \EndFor
    \EndFor

    \Repeat
        \For{$d \in D$}
            \For{$w \in W_d$}
                \State $z_{old} \set Z[d,w]$
                \State $z_{new} \set \text{gibbs-sample}(z_{old}, ndz, nwz[w], nz, \alpha, \beta)$
                \State $Z[d,w] \set z_{new}$
                \State $ndz[d,z_{old}] \set ndz[d,z_{old}] - 1$
                \State $nwz[w,z_{old}] \set nwz[w,z_{old}] - 1$
                \State $nz[z_{old}] \set nz[z_{old}] - 1$

                \State $ndz[d,z_{new}] \set ndz[d,z_{new}] + 1$
                \State $nwz[w,z_{new}] \set nwz[w,z_{new}] + 1$
                \State $nz[z_{new}] \set nz[z_{new}] + 1$
            \EndFor
        \EndFor
    \Until {Stop condition is satisfied}

    \State $P(z|d) \set \text{normalize}(ndz, \alpha)$
    \State $P(z|w) \set \text{normalize}(nwz, \beta)$
\end{algorithmic}
\end{algorithm}
    

\begin{description}
    \item[Parallelism] The inner loop is sequential.
        The outer loop is data-parallel and model averaging is used.
\end{description}

\subsection{Implementation as User-Defined Function}

Algorithm\texttt{gibbs-lda} is implemented as the user-defined function \texttt{lda\_train}.

\begin{center}
    \begin{tabularx}{\textwidth}{rlXl}
        \toprule%
        \textbf{Name} & \textbf{Description} & \textbf{Type}
        \\\otoprule

        In &
        \texttt{data\_table} &
        Table containing the training dataset &
        Relation
        \\\midrule

       In &
        \texttt{voc\_size} &
        Size of vocabulary &
        Integer
        \\\midrule

        In &
        \texttt{topic\_num} &
        Number of topics &
        Integer
        \\\midrule

        In &
        \texttt{iter\_num} &
        Number of iterations &
        Integer
        \\\midrule
    
        In &
        \texttt{alpha} &
        Prior on per-document topic distribution &
        Double
        \\\midrule

        In &
        \texttt{beta} &
        Prior on per-word topic distribution &
        Double
        \\\midrule

        Out &
        \texttt{model\_table} &
        Table containing the model information &
        Relation
        \\\midrule

        Out &
        \texttt{output\_data\_table} &
        Table containing the per-document topic counts and topic assignments &
        Relation
        \\\bottomrule
     \end{tabularx}
\end{center}

Internally, two work tables are used alternately in the iterative Gibbs sampling process, one as input, another as output. The key part of an iteration is implemented essentially using the following SQL:

\begin{sql}[emph={work_table_out,work_table_in,__newplda_gibbs_sample,__newplda_count_topic_agg,model}]
    INSERT INTO work_table_out
    SELECT  
        distid, 
        docid, 
        wordcount, 
        words, 
        counts,  
        madlib.__newplda_gibbs_sample(
            words, 
            counts, 
            doc_topic, 
            model, 
            alpha, 
            beta, 
            voc_size, 
            topic_num)
    FROM
    (
        SELECT
            distid, 
            docid, 
            wordcount, 
            words,
            counts, 
            doc_topic, 
            model 
        FROM
        (
                SELECT
                    madlib.__newplda_count_topic_agg(
                        words, 
                        counts,
                        doc_topic[topic_num + 1:array_upper(doc_topic, 1)] AS topic_assignment, 
                        voc_size, 
                        topic_num) model 
                FROM 
                    work_table_in 
        ) t1 
        JOIN 
        work_table_in
    ) t2
\end{sql}

Note that within the \texttt{madlib.\_\_newplda\_gibbs\_sample} function, the \texttt{model} parameter will be read in the first invocation and stored in the memory. In the incoming invocations within the same query, the parameter will be ignored. In this way, the model can be updated by an invocation and the updated model can be transferred to the next invocation.

The above SQL can be further rewritten to eliminate the data redundancy and reduce the overhead of joining operation, and thus improve the overall performance. This is very useful when the product of $voc\_size \times topic\_num$ is very large. See below for the rewritten SQL:

\begin{sql}[emph={work_table_out,work_table_in,__newplda_gibbs_sample,__newplda_count_topic_agg,model}]
    INSERT INTO work_table_out
    SELECT  
        distid, 
        docid, 
        wordcount, 
        words, 
        counts,  
        madlib.__newplda_gibbs_sample(
            words, 
            counts, 
            doc_topic, 
            model, 
            alpha, 
            beta, 
            voc_size, 
            topic_num)    
    FROM
    (
        SELECT
            dcz.distid, 
            dcz.docid, 
            dcz.wordcount, 
            dcz.words,
            dcz.counts, 
            dcz.doc_topic, 
            chunk.model 
        FROM
        (
            SELECT
                distid, docid, model
            FROM
            (
                SELECT
                    madlib.__newplda_count_topic_agg(
                        words, 
                        counts,
                        doc_topic[topic_num + 1:array_upper(doc_topic, 1)] AS topic_assignment, 
                        voc_size, 
                        topic_num) model 
                FROM
                    work_table_in 
            ) t1,
            (
                SELECT
                    distid, 
                    min(docid) docid 
                FROM
                    work_table_in 
                GROUP BY distid
            ) t2 -- One row per-segment
        ) chunk -- Ideally only one row per-segment
        RIGHT JOIN work_table_in dcz 
        ON (dcz.distid = chunk.distid AND dcz.docid = chunk.docid)
        ORDER BY distid, docid -- Local data manipulation, no data redistribution
    ) joined -- Ideally only one row per-segment has the fully joined data 
\end{sql}

% When using TeXShop on the Mac, let it know the root document. The following must be one of the first 20 lines.
!TEX root = ../design.tex

\chapter[Regression]{Regression}
\begin{moduleinfo}
\item[History]
	\begin{modulehistory}
		\item[v0.1] Initial version, including background of regression
	\end{modulehistory}
\end{moduleinfo}

% Abstract. What is the problem we want to solve?

Regression analysis is a statistical tool for the investigation of
relationships between variables. Usually, the investigator seeks to ascertain
the causal effect of one variable upon another—the effect of a price increase
upon demand, for example, or the effect of changes in the money supply upon the
inflation rate. More specifically, regression analysis helps one understand how
the typical value of the dependent variable changes when any one of the
independent variables is varied, while the other independent variables are held
fixed.

Regression models involve the following variables:
\begin{enumerate}
    \item The unknown parameters, denoted as $\beta$, which may represent a scalar or a vector.
    \item The independent variables, $x$
    \item The dependent variables, $y$
\end{enumerate}

\section{Linear Methods for Regression} % (fold)
\label{sub:linear_methods_for_regression}

% subsection linear_methods_for_regression (end)

\section{Regularization} % (fold)
\label{sub:regularization}

Usually, $y$ is the result of measurements contaminated by small errors
(noise). Frequently, ill-conditioned or singular systems also arise in the iterative solution of nonlinear systems or optimization problems. In all such situations, the vector $x = {A}^{-1}y$ (or in the full rank overdetermined
case $A^+ y$, with the pseudo inverse $A^+ = (A^T A)^{−1}A^T X)$, if it exists at all, is usually a meaningless bad approximation to x.

Regularization techniques are needed to obtain meaningful solution estimates 
for such ill-posed problems, and in particular when the number of parameters 
is larger than the number of available measurements, so that standard least 
squares techniques break down.

\subsection{Linear Ridge Regression}
Ridge regression is the most commonly used method of regularization of
ill-posed problems. Mathematically, it seeks to minimize

\begin{equation}
Q\left(\vec{w},w_0;\lambda\right)\equiv \min_{\vec{w},w_0}\left[ \frac{1}{2N} \sum_{i=1}^{N} \left( y_i - w_0 -
    \vec{w} \cdot \vec{x}_i \right)^2
  +\frac{\lambda}{2}\|\vec{w}\|_2^2 \right]\ ,
\end{equation}
for a given value of $\lambda$, where $\vec{w}$ and $w_0$ are the fitting coefficients, and $\lambda$
is a non-negative regularization parameter. $\vec{w}$ is a vector in
$d$ dimensional space, and
\begin{equation}
\|\vec{w}\|_2^2 = \sum_{j=1}^{d}w_j^2 = \vec{w}^T\vec{w}\ .
\end{equation}
When $\lambda = 0$, $Q$ is
the mean squared error of the fitting.

The intercept term $w_0$ is not regularized, because this term is
fully decided by the mean values of $y_i$ and $\vec{x}_i$ and the
values of $\vec{w}$, and does not affect the model's complexity.

$Q\left(\vec{w},w_0;\lambda)$ is a quadratic function of $\vec{w}$ and
  $w_0$, and thus can be solved analytically
\begin{equation}
\vec{w}_{ridge}=\left(\lambda\vec{I}_d +
  \vec{X}^T\vec{X}\right)^{-1}\vec{X}^T\vec{y}\ .
\end{equation}
By using the available Newton method (Sec. 6.2.4), the above quantity can be easily
calculated from one single step of the Newton method.

Many packages for Ridge regularization actually regularize the fitting
coefficients not for the fitting model for the original data but for
the data that has be scaled. MADlib also provides this option. When
the normalization parameter is set to be True, which is by default
False, the data will be first converted to the following before
applying the Ridge regularization.
\begin{equation}
  x_i' \leftarrow \frac{x_i - \langle x_i \rangle}{\langle (x_i -
    \langle x_i \rangle)^2\rangle} \ ,
\end{equation}
\begin{equation}
y_i \leftarrow y_i - \langle y_i \rangle \ ,
\end{equation}
where $\langle \cdot \rangle = \sum_{i=1}^{N} \cdot / N$.

Note that Ridge regressions for scaled data and un-scaled data are not equivalent.

\subsection{Elastic Net Regularization} % (fold)
\label{ssub:elastic_net_regularization}
As a continuous shrinkage method, ridge regression achieves its better prediction performance through a bias–variance trade-off. However, ridge regression cannot produce a parsimonious model, for it always keeps all the predictors in the model~\cite{zou2005}. Best subset selection in contrast produces a sparse model, but it is extremely variable because of its inherent discreteness. 

A promising technique called the lasso was proposed by Tibshirani (1996). The 
lasso is a penalized least squares method imposing an L1-penalty on the 
regression coefficients. Owing to the nature of the L1-penalty, the lasso does 
both continuous shrinkage and automatic variable selection simultaneously.
 
Although the lasso has shown success in many situations, it has some 
limitations. Consider the following three scenarios: 
\begin{enumerate}
    \item In the Number of features ($p$) >> Number of observations ($n$) case, the lasso selects at most $n$ variables before it saturates, because of the nature of the convex optimization problem. This seems to be a limiting feature for a variable selection method. Moreover, the lasso is not well defined unless the bound on the $L_1$-norm of the coefficients is smaller than a certain value.
    \item If there is a group of variables among which the pairwise correlations are very high, then the lasso tends to select only one variable from the group and does not care which one is selected.
    \item For usual n>p situations, if there are high correlations between predictors, it has been empirically observed that the prediction performance of the lasso is dominated by ridge regression. 
\end{enumerate}

These scenarios make lasso an inappropriate variable selection method in some situations. 

Hui Zou and Trevor Hastie [42] introduce a new regularization 
technique called the 'elastic net'. Similar to the lasso, the elastic net 
simultaneously does automatic variable selection and continuous
shrinkage, and it can select groups of correlated variables. It is like a 
stretchable fishing net that retains `all the big fish'.

The elastic net regularization minimizes the following target function
\begin{equation} \label{eq:target}
\min_{\vec{w} \in R^N}L(\vec{w}) + \lambda \left[\frac{1-\alpha}{2}\|\vec{w}\|_2^2 +
  \lambda\alpha \|\vec{w}\|_1\right]\ ,
\end{equation}  
where $\|\vec{w}\|_1 = \sum_{i=1}^N|w_i|$ and $N$ is the number of features.

For the elastic net regularization on linear models, 
\begin{equation}
L(\vec{w}) = \frac{1}{2M}\sum_{m=1}^M\left(y_m - w_0 - \vec{w} \cdot
  \vec{x}_m\right)^2\ ,
\end{equation}
where the sum is over all observations and $M$ is the total number of
observation.

For the elastic net regularization on logistic models,
\begin{equation}
L(\vec{w}) = \sum_{m=1}^M\left[y_m \log\left(1 + e^{-(w_0 +
      \vec{w}\cdot\vec{x}_m)}\right) + (1-y_m) \log\left(1 + e^{w_0 +
      \vec{w}\cdot\vec{x}_m}\right)\right]\ ,
\end{equation}
where $y_m \in \{0,1\}$.

\subsubsection{Optimizer Algorithms}
Right now, we support two algorithms for optimizer. The default one is
FISTA, and the other is IGD.

\paragraph{FISTA}

Fast Iterative Shrinkage Thresholding Algorithm (FISTA) with {\bf
  backtracking step size} [4]:
\vspace{0.2in}
\hline
\vspace{0.2in}
{\bf Step $0$}: Choose $\delta>0$ and $\eta > 1$, and
$\vec{w}^{(0)} \in R^N$. Set $\vec{v}^{(1)}=\vec{w}^{(0)}$ and
$t_1=1$.

{\bf Step $k$}: ($k \ge 1$) Find the smallest nonnegative integers
$i_k$ such that with $\bar{\delta} = \delta_{k-1}/\eta^{i_k-1}$
\begin{equation}
F(p_{\bar{\delta}}(\vec{v}^{(k)})) \le
Q_{\bar{\delta}}(p_{\bar{\delta}}(\vec{v}^{(k)}), \vec{v}^{k})\ .
\end{equation}
Set $\delta_k = \delta_{k-1}/\eta^{i_k-1}$ and compute
\begin{equation}
\vec{w}^{(k)}  =  p_{\delta_k}\left(\vec{v}^{(k)}\right)\ , 
\end{equation}
\begin{equation}
t_{k+1} = \frac{1 + \sqrt{1 + 4t_k^2}}{2}\ ,
\end{equation}
\begin{equation}
\vec{v}^{(k+1)} = \vec{w}^{(k)} +
\frac{t_k-1}{t_{k+1}}\left(\vec{w}^{(k)} - \vec{w}^{(k-1)}\right)\ .
\end{equation}
\vspace{0.2in}
\hline
\vspace{0.2in}
Here,
\begin{equation}
F(\vec{w}) = f(\vec{w}) + g(\vec{w})\ ,
\end{equation}
where $f(\vec{w})$ is the differentiable part of
Eq. (\ref{eq:target}) and $g(\vec{w})$ is the non-differentiable part. For linear models, 
\begin{equation}
f(\vec{w}) = \frac{1}{2M}\sum_{m=1}^M\left(y_m - w_0 - \vec{w} \cdot
  \vec{x}_m\right)^2 + \frac{\lambda(1-\alpha)}{2}\|\vec{w}\|_2^2\ ,
\end{equation}
and for logistic models,
\begin{equation}
f(\vec{w}) = \sum_{m=1}^M\left[y_m \log\left(1 + e^{-(w_0 +
      \vec{w}\cdot\vec{x}_m)}\right) + (1-y_m) \log\left(1 + e^{w_0 +
      \vec{w}\cdot\vec{x}_m}\right)\right] + \frac{\lambda(1-\alpha)}{2}\|\vec{w}\|_2^2\ .
\end{equation}
And for both types of models, 
\begin{equation}
g(\vec{w}) = \lambda\alpha\sum_{i=1}^N|w_i|\ .
\end{equation}

And 
\begin{equation}
Q_{\delta}(\vec{a}, \vec{b}) := f(\vec{b}) + \langle \vec{a} -
\vec{b}, \nabla f(\vec{b})\rangle + 
\frac{1}{2\delta}\|\vec{a} - \vec{b}\|^2 + g(\vec{a})\ ,
\end{equation}
where $\langle\cdot\rangle$ is just the usual vector dot product.

And the proxy function is defined as
\begin{equation}
p_\delta (\vec{v}) := \underset{\vec{w}}{\operatorname{arg\,min}} \left[g(\vec{w}) +
  \frac{1}{2\delta}\left\|\vec{w} - \left(\vec{v} - \delta\,\nabla f(\vec{v})\right)\right\|^2  \right]
\end{equation}
For our case, where $g(\vec{w}) = \lambda\alpha\sum_{i=1}^N|w_i|$, the
proxy function is simply equal to the soft-thresholding function
\begin{equation}
p_\delta (v_i) = \left\{ \begin{array}{ll}
v_i - \lambda\alpha\delta u_i\ , \quad  & \mbox{if } v_i > \lambda\alpha\delta
u_i \\
0\ , \quad & \mbox{otherwise} \\
v_i + \lambda\alpha\delta u_i\ , \quad & \mbox{if } v_i < - \lambda\alpha\delta u_i
\end{array}
\right.
\end{equation}
where 
\begin{equation}
\vec{u} = \vec{v} - \delta\,\nabla f(\vec{v})\ .
\end{equation}

{\bf Active set method} is used in our implementation for FISTA to
speed up the computation. Considerable speedup is obtained by
organizing the iterations around the active set of features— those
with nonzero coefficients. After a complete cycle through all the
variables, we iterate on only the active set till convergence. If
another complete cycle does not change the active set, we are done,
otherwise the process is repeated.

{\bf Warm-up method} is also used to speed up the computation. When
the option parameter $warmup$ is set to be $True$, a series of lambda
values, which is strictly descent and ends at the lambda value that
the user wants to calculate, will be used. The larger lambda gives
very sparse solution, and the sparse solution again is used as the
initial guess for the next lambda's solution, which will speed up the
computation for the next lambda. For larger data sets, this can
sometimes accelerate the whole computation and might be faster than
computation on only one lambda value.

{\bf Note:} Our implementation is a little bit different from the
original FISTA. In the original FISTA, during the backtracking
procedure, the algorithm is searching for a non-negative integer $i_k$
and the new step size is $\delta_k = \delta_{k-1}/\eta^{i_k}$. Thus
the step size is non-increasing. Here, we allow the step size to
increase by using $\delta_k = \delta_{k-1}/\eta^{i_k-1}$ so that
larger step sizes can be tried by the algorithm. Tests show that this
is actually faster.

\paragraph{IGD}

The Incremental Gradient Descent (IGD) algorithm is a stochastic
algorithm by its nature. So it is difficult to get sparse
solutions. What we implemented is Stochastic Mirror Descent Algorithm
made Sparse (SMIDAS). The inverse p-form link function is used
\begin{equation} \label{eq:f}
h_j^{-1}(\vec{\theta}) = \frac{\mbox{sign}(\theta_j)\vert \theta_j
  \vert^{p-1}}{\|\theta\|_p^{p-2}}\ ,
\end{equation}
where
\begin{equation}
\|\theta\|_p = \left(\sum_j \vert \theta \vert ^p\right)^{1/p}\ ,
\end{equation}
and $\mbox{sign}(0) = 0$.
\vspace{0.2in}
\hline
\vspace{0.2in}
Choose step size $\delta > 0$.

Let $p = 2\log d$ and let $h^{-1}$ be as in Eq. (\ref{eq:f})

Let $\vec{\theta} = \vec{0}$

{\bf for} $k = 1,2,\dots$

\quad $\vec{w} = h^{-1}(\vec{\theta})$

\quad $\vec{v} = \nabla f(\vec{w})$

\quad $\tilde{\vec{\theta}} = \theta - \delta \, \vec{v}$

\quad $\forall j, \theta_j = \mbox{sign}(\tilde{\theta}_j)\max (0,
\vert \tilde{\theta}_j \vert- \lambda\alpha\delta)$
\vspace{0.2in}
\hline
\vspace{0.2in}

The resulting fitting coefficients of this algorithm is not really
sparse, but the values are very small (usually $< 10^{15}$), which can
be safely set to be zero after filtering with a threshold.

This is done as the following: (1) multiply each coefficient with the
standard deviation of the corresponding feature (2) compute the
average of absolute values of re-scaled coefficients (3) divide each
rescaled coefficients with the average, and if the resulting absolute
value is smaller than <em>threshold</em>, set the original coefficient
to be zero.

IGD is in nature a sequential algorithm, and when running in a
distributed way, each segment of the data runs its own SGD model, and
the models are averaged to get a model for each iteration. This
average might slow down the convergence speed, although we acquire the
ability to process large data set on multiple machines. So this
algorithm provides an option <em>parallel</em> to let the user choose
whether to do parallel computation.

IGD also implements the {\bf warm-up method}.

{\bf Stopping Criteria} Both {\bf FISTA} and {\bf IGD} compute the average difference
between the coefficients of two consecutive iterations, and if the
difference is smaller than <em>tolerance</em> or the iteration number
is larger than <em>max_iter</em>, the computation stops.


% subsubsection elastic_net_regularization (end)
% subsection regularization (end)


\printbibliography

\end{document}
